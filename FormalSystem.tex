% Created 2016-09-15 tor 14:09
\documentclass[11pt]{article}
\usepackage[utf8]{inputenc}
\usepackage[T1]{fontenc}
\usepackage{fixltx2e}
\usepackage{graphicx}
\usepackage{grffile}
\usepackage{longtable}
\usepackage{wrapfig}
\usepackage{rotating}
\usepackage[normalem]{ulem}
\usepackage{amsmath}
\usepackage{textcomp}
\usepackage{amssymb}
\usepackage{capt-of}
\usepackage{hyperref}
\usepackage{mathpartir}
\usepackage{fontspec}
\usepackage{unicode-math}
\setmainfont[]{DejaVu Sans}
\newcommand{\case}[2]{\mathsf{case} #1 \mathsf{of} \{#2\}^m_{k=1}}
\newcommand{\inl}{\mathsf{inl} }
\newcommand{\inr}{\mathsf{inr} }
\newcommand{\flet}[1][]{\mathsf{let}_{#1} }
\newcommand{\fin}{ \mathsf{in} }
\newcommand{\susp}[1]{⟦#1⟧}
\author{Jean-Philippe Bernardy and Arnaud Spiwack}
\date{\today}
\title{Linear and unrestricted at the same time}
\hypersetup{
 pdfauthor={Jean-Philippe Bernardy and Arnaud Spiwack},
 pdftitle={Linear and unrestricted at the same time},
 pdfkeywords={},
 pdfsubject={},
 pdfcreator={Emacs 24.5.2 (Org mode 8.3.5)}, 
 pdflang={English}}
\begin{document}

\maketitle
\tableofcontents

In this note we give an overview of a lambda calculus augmented with
weights on variable (bindings). We argue briefly that the calculus is
a generalisation of both:
\begin{itemize}
\item the simply-typed lambda calculus and
\item the linear lambda calculus
\end{itemize}

We then provide an operational semantics which demonstrates how memory
management of linear values can be simplified.

\section{Statics}
\label{sec:orgheadline8}
\subsection{Weights}
\label{sec:orgheadline1}

The set of weights is the two-element set \{1,ω\}. Intuitively, 1
represents \textbf{exactly one} instance, while ω represents any number of
instances (including zero).  The metasyntactic variables \(π\) and \(ρ\)
range over this set. Weights are equipped with + and ∙ operations:


\begin{align*}
1 · ω = ω \\
ω · 1 = ω \\
ω · ω = ω \\
1 · 1 = 1
\end{align*}

\begin{align*}
1 + ω &= ω \\
ω + 1 &= ω \\
1 + 1 &= ω \\
ω + ω &= ω
\end{align*}

We equip weights we a partial order (written \(· ≤ ·\)) such that \(1 ≤ ω\).

\emph{(An alternative which may be worth pursuing is to choose the set of
weights to be \(ℕ∪{ω}\), in which case \(1+1=2\).)}

\subsection{Types}
\label{sec:orgheadline2}

Let us consider a variant of simply typed λ-calculus where the types
are:

\begin{itemize}
\item a weighted arrow: \(A →_π B\) (you can get a \(B\) if you can provide a
quantity \(π\) of \(A\))
\begin{itemize}
\item \(A →_ω B\) is the intuitionistic arrow \(A → B\)
\item \(A →_1 B\) is the linear arrow \(A ⊸ B\)
\end{itemize}
\item weighted ADTs.

\begin{align*}
\mathsf{data} D = (c_k  π₁ A₁  …  π_{n_k} A_{n_k})^m_{k=1}
\end{align*}
(\(D\) has \(m\) constructors \(c_k\), for \(k ∈ 1…m\)).

Note the following special cases:
\begin{itemize}
\item The type data Pair a b = Pair ωa ωb is isomorphic to the intuitionistic product (a×b)
\item The type data Tensor a b = Pair 1a 1b is isomorphic to the linear tensor product (a⊗b)
\item The type data Bang a = Box ωa is isomorphic to the exponential (!a)
\end{itemize}
\end{itemize}

Syntax of types and type declarations ($c$ ranges over constructor names):
\begin{align*}
  W &::= πA &\text{weight-annotated type}\\
  γ &::= c  \vec{W}&\text{constructor declaration}\\
  decl &::= \mathsf{data } D = \vec{γ}&\text{data type declaration}\\
  A,B &::=\\
      & |  A →_π B &\text{function type}\\
      & |  D &\text{data type}
\end{align*}

\subsection{Terms}
\label{sec:orgheadline3}

Syntax for terms:

\begin{align*}
e,s,t,u & ::= \\
    & |  x & \text{variable} \\
    & |  λx. t & \text{abstraction} \\
    & |  t s & \text{application} \\
    & |  c t₁ … t_n & \text{data construction} \\
    & |  \case t {c_k  x₁ … x_{n_k} → u_k}  & \text{case} \\
    & |  \flet x =_{π₁} … x =_{π_n} t_n \fin u & \text{let}
\end{align*}

\subsection{Contexts}
\label{sec:orgheadline6}

Every variable binding in a context is equipped with a weight,
additionally to the type:

\begin{align*}
  Γ,Δ & ::=\\
    & |  x :_π A, Γ & \text{weight-annotated binder} \\
    & |     & \text {empty context}
\end{align*}

\subsubsection{Context addition}
\label{sec:orgheadline4}

We define a context addition, written \(Γ+Δ\), as follows.

\begin{itemize}
\item When a variable appears on both input contexts, we add the weights in
the result:

\((x :_ρ A,Γ) + (x :_π A,Δ) = x :_{ρ+π} A, (Γ+Δ)\)

\item otherwise we propagate:

\((x :_ρ A,Γ) + Δ = x :_ρ A, Γ+Δ\)   \hfill   \(x ∉ Δ\)
\end{itemize}


\subsubsection{Context scaling}
\label{sec:orgheadline5}

Contexts can be scaled by a weight:

\begin{displaymath}
ρ(x :_π A, Γ) =  x :_{πρ} A, ρΓ
\end{displaymath}

\subsection{Typing rules}
\label{sec:orgheadline7}

We have a weighted typing judgement \(Γ ⊢ x:_π A\), inductively defined by the following rules.

\begin{mathpar}
\inferrule{ }{ωΓ + x :_ρ A ⊢ x :_ρ A}\text{variable}

\inferrule{Γ, x :_{πρ} A  ⊢   t :_ρ B}
          {Γ ⊢ λx. t  :_ρ  A  →_π  B}\text{abs}

\inferrule{Γ ⊢ t :_ρ  A →_π B  \\   Δ ⊢ u :_{πρ} A}
          {Γ+Δ ⊢ t u  :_ρ  B}\text{app}

\inferrule{Δᵢ ⊢ tᵢ :_{πᵢρ} Aᵢ \\ \text {$c_k$ is a constructor of $D$ with arguments $Aᵢ$ and weights $πᵢ$}}
          {\sum_i Δᵢ ⊢ c_k  t₁ … t_n :_ρ  D}\text{constructor}

\inferrule{Γ   ⊢  t  :_{ρ} A  \\ Δ, x₁:_{πᵢρ} Aᵢ, …, x_{n_k}:_{π_{n_k}ρ} A_{n_k} ⊢ u_k :_ρ C \\ \text{with each $c_k$ as above}}
          {Γ+Δ ⊢ \case t {c_k  x₁ … x_{n_k} → u_k} :_ρ C}\text{case}

\inferrule{Γᵢ   ⊢  tᵢ  :_{πᵢρ} Aᵢ  \\ Δ, x₁:_{π₁ρ} A₁ …  x_n:_{π_{nρ}} A_n ⊢ u :_ρ C }
          {\sum_i Γᵢ+Δ ⊢ \flet x =_{π₁} t₁  …  x =_{π_n} t_n  \fin u :_ρ C}\text{let}


\end{mathpar}

Lemmas:
\begin{enumerate}
\item \(πΓ ⊢ t:_{πρ} A\) if \(Γ ⊢ t:_ρ A\).
\item erasure of weights preserves typing
\end{enumerate}


\section{Dynamics}
\label{sec:orgheadline16}
\subsection{(Extended) Launchbury semantics}
\label{sec:orgheadline11}

\subsubsection{Translation}
\label{sec:orgheadline9}
As Launchbury, we translate from terms to terms where values are
always bound to variables. However, we must do so on typed terms in
general, because we need the full info about weights.


Nonetheless, a sub-part of the syntax does not need the weights, and
we just write the translation rules in compact form, as Launchbury:
\begin{align*}
(λx. t)^* &= λx. (t)^* \\
x^*       &= x \\
(t₁ x )^* &= (t₁)^* x \\
(\case y {c_k  x₁ … x_{n_k} → u_k})^* &= \case y {c_k  x₁ … x_{n_k} → (u_k)^*} \\
\end{align*}

For the application case where e₂ is not a variable, we need the
weight information:
\begin{mathpar}
\left(\inferrule{Γ ⊢ t :_ρ  A →_π B  \\   Δ ⊢ u :_{πρ} A}
          {Γ+Δ ⊢ t u  :_ρ  B}\right)^* =
  \flet y =_{πρ} (u)^* \fin (t)^* y
\end{mathpar}

Constructors are treated similarly:
\begin{mathpar}
\left(\inferrule{Δᵢ ⊢ tᵢ :_{πᵢρ} Aᵢ}
     {\sum_i Δᵢ ⊢ c_k  t₁ … t_n :_ρ  D}\right)^* =

     \flet x₁ =_{π_1ρ} (t₁)^*,…, x_n =_{π_nρ} (t_n)^* \fin c_k x₁ … x_n
\end{mathpar}
As for application, the translation does not need to bind arguments that are already
variables. (We omit the simplified rules.)


In case, we bind the intermediate results to a variable of the
appropriate linearity:
\begin{mathpar}
\left(\inferrule{Γ   ⊢  t  :_{ρ} A  \\ Δ, x₁:_{πᵢρ} Aᵢ, …, x_{n_k}:_{π_{n_k}ρ} A_n ⊢ u_k :_ρ C}
          {Γ+Δ ⊢ \case t {c_k  x₁ … x_{n_k} → u_k} :_ρ C}\right)^* =

          \flet y =_ρ (t)^* \fin \case y {c_k  x₁ … x_{n_k} → (u_k)^*}
\end{mathpar}

In $\flet$, the weights must be scaled too:
\begin{mathpar}
\left(\inferrule{Γᵢ   ⊢  tᵢ  :_{πᵢρ} Aᵢ  \\ Δ, x₁:_{π₁ρ} A₁ …  x_n:_{π_{n}ρ} A_n ⊢  \\ Δ ⊢ u :_ρ C }
          {\sum_i Γᵢ+Δ ⊢ \flet x =_{π₁} t₁  …  x =_{π_n} t_n  \fin u :_ρ C}\right)^* =

          \flet x₁ =_{ρπ_1} (t₁)^*,…, x_n =_{ρπ_n} (t_n)^* \fin (u)^* \\
\end{mathpar}


\subsubsection{Natural semantics}
\label{sec:orgheadline10}
Compared to Launchbury, we add a linear heap, ranged by Φ,Ψ,Ξ.  (GC'ed
heap is ranged by Γ,Δ,Θ)


\begin{mathpar}
\inferrule{ }{Γ;Φ : λx. e ⇓ Γ;Φ : λx. e}\text{abs}


\inferrule{Γ;Φ : e ⇓ Δ;Ψ : λy.e'   Δ;Ψ : e'[x/y] ⇓ Θ;Ξ : z}
           {Γ;Ψ : e x ⇓ Θ;Ξ : z} \text{application}

\inferrule{Γ;Φ : e ⇓ Δ;Ψ : z}{(Γ,x ↦ e;Φ) : x ⇓ (Δ;x ↦ z;Ψ) : z}\text{unrestricted variable}


     \inferrule{Γ;Φ : e ⇓ Δ;Ψ : z}
{(Γ;Φ,x ↦ e) : x ⇓ Δ;Ψ : z}\text{linear variable}


\inferrule{(Γ,x_i ↦ e_i;Φ,x_j ↦ e_j) : e ⇓ Δ;Ψ : z}
{Γ,Φ : \flet x₁ =_{π₁} e₁ … x_n =_{π_n} e_n \fin e ⇓ Δ;Ψ : z}\text{let}

\text{where $i$ ranges over the subset of indices such that $πᵢ$ = ω,}

\text{and $j$ ranges over the subset such that $π_j$ = 1.}


\inferrule{ }{Γ;Φ : c  x₁ … x_n ⇓ Γ;Φ : c  x₁ … x_n}\text{constructor}


\inferrule{Γ;Φ:e ⇓ Δ,Ψ:c_k  x₁ … x_n    Δ,Ψ : e_k[xᵢ/yᵢ] ⇓ Θ,Ξ : z}
   {Γ;Φ : \case e {c_k  y₁ … y_n ↦ e_k } ⇓ Θ,Ξ : z}\text{case}
\end{mathpar}

Remark: the \emph{variable} rule finds out if things come from linear
or unrestricted heap dynamically. This behaviour allows a linear
occurence of a variable to work in an unrestricted contexts; thus
justifying the $1 + ω = ω$ rule.

Theorem: the unrestricted heap contains no references to the linear
heap.

\subsubsection{Will there be anything in the linear heap?}

One may believe that if the top-level result of the program have
weight ω, then all my inputs must have weight ω (otherwise we could
produce something out of nothing). This line of thought is incorrect:
data types can be copied. For any data type $D$, we will have a
function $D ⊸ Bang D$ (this may be even efficiently implemented by
copying a single pointer --- for example if we have a single array, or
a notion of compact region).

Thus, at any point where we have an intermediate result comprised of
data only, we may switch to use the linear heap. In a second phase,
this data may then be moved to the GC heap and used for general
consumption.


\subsection{{\bfseries\sffamily TODO} KAM}
\label{sec:orgheadline14}

Skip this section: it is not adapted yet to the syntax above.

The dynamic semantics is given as a variant of Krivine's abstract
machine, modified to support two environments. One environment is
unrestricted, while the other is linear.

\begin{itemize}
\item As in Krivine's original, the abstract machine has a call-by-name
evaluation strategy, not true laziness (that is, sharing of the
intermediate results). This is sufficient to demonstrate the
principles of linear logic memory management.

\item We assume that the input terms are annotated with:
\begin{itemize}
\item weights (for variables)
\item how the context is split (for terms where this happens)
\end{itemize}

\item The machine runs the same way in ω or 1 context. This means that we
have 'weight polymorphism' at runtime. (The same code can run in
linear or unrestricted environment.)
\end{itemize}

\begin{align*}
state       &::= (t,σ,φ,ψ) & \text{machine state} \\
ξ     &::= (t,φ,ψ) \mid \susp{t,φ,ψ} & \text{closure} \\
σ       &::= [] \mid ξ:σ & \text{stack} \\
φ &::= [] \mid (x↦ξ,φ) & \text{environment} \\
ψ &::= [] \mid (x↦ξ,φ) & \text{linear environment}
\end{align*}

\begin{itemize}
\item We write \(ψ/Γ\) for the environment restricted to the variables and
weights in \(Γ\). We have the following property: if \(Γ = Δ₁+Δ₂\) and
\(ψ : Γ\), then \(ψ = ψ/Δ₁ ⊎ ψ/Δ₂\) (because \(ψ\) is a linear
environment, \(ψ/Δ₁\) and \(ψ/Δ₂\) have disjoint support).
\end{itemize}

\hspace{-4cm}\begin{minipage}{\textwidth}
\begin{align*}
(u_Γ t_Δ,σ,φ,ψ)                                                  &⟶ (u,((t,φ,ψ/Δ):σ),φ,ψ/Γ)\\
(λx_ω.t,ξ:σ,φ,ψ)                                                    &⟶ (t,σ,((x↦ξ),φ),ψ)\\
(λx_1.t,ξ:σ,φ,ψ)                                                    &⟶ (t,σ,φ,(x↦ξ),ψ)\\
(x_1,σ,\_,[x↦(t,φ,ψ)])                                               &⟶ (t,σ,φ,ψ)\\
(x_ω,σ,(x↦(t,φ,ψ):\_),[])                                            &⟶ (t,σ,φ,ψ)\\
(\inl t,\susp{\case □ {\inl x → u; \inr y → u'}:σ,φ',ψ'},φ,ψ) &⟶ (u,σ,φ',(x↦(t,φ,ψ)):ψ')\\
(\case t {\inl x → u; \inr y → u'},σ,φ,ψ)                        &⟶ (t, \susp{\case □ {\inl x → u; \inr y → u'},φ,ψ/Γ}:σ,φ,ψ/Δ)\\
(\flet (x_π,y_ρ) = t \fin u,σ,φ,ψ)                                       &⟶ (t, \flet (x_π,y_ρ) = □ \fin u,φ,ψ/Γ):σ,φ,ψ/Δ)\\
((t_Γ,u_Δ),\susp{\flet (x_π,y_ρ) = □ \fin v:σ,φ',ψ'},φ,ψ')         &⟶ (v,σ,φ', (x↦(t,φ,ψ/Γ)),(y↦(u,φ,ψ/Γ)),ψ')
\end{align*}
\end{minipage}

\subsubsection{Theorems}
\label{sec:orgheadline12}

\begin{itemize}
\item If \(Γ ⊢ t :_ρ A\), then \((t,[],[],[]) ⟶^* (u,\_,\_,\_)\) where \(u\) is in whnf.
\item No part of the linear environment is ever duplicated (or shared).
\end{itemize}

\subsubsection{Remarks}
\label{sec:orgheadline13}

\begin{itemize}
\item In the reduction \((u_Γ t_Δ,σ,φ,ψ)\), one may fear that we're
putting a linear part of the environment \(Δ\) in a closure, which will
end up being shared afterwards. The system is engineered so that
this case cannot occur. Indeed
\begin{itemize}
\item when the weight of \(t\) is \(ω\), then \(Δ\) contains only \(ω\) weights; the
linear part of \(Δ\) is empty, so is \(ψ/Δ\):

\((u_Γ t_{ωΔ},σ,φ,ψ)          ⟶ (u,((t,φ,[]):σ),φ,ψ)\)

\item when the weight of \(t\) is 1, then the closure will only be consumed
by a linear λ --- it will not be duplicated.
\end{itemize}

\item Consequently, for every closure in the unrestricted environment, the
linear environment embedded in the closure is empty.
\end{itemize}


\subsection{Examples of simple programs and their types}

\begin{align*}
λx. λy. x & : A ⊸ B → A \\
          & : A → B → A \\
λ(f,x). f x & : \{f :_1 A ⊸ B, x :_1 A\} ⊸ B & \text{(using a row type for concision)}\\
            & : \{f :_1 A → B, x :_ω A\} ⊸ B \\
            & : \{f :_1 A →_π B, x :_π A\} ⊸ B & \text{most general type} \\ 
λf. λg. λx. f (g x) & : (B → C) → (A → B) → A → C \\
                    & : (B → C) ⊸ (A → B) → A → C & \text {always better: comp uses $f$  only once} \\
                    & : (B ⊸ C) ⊸ (A → B) ⊸ A → C & \text {if $f$ is linear, then $g$ is used only once} \\
                    & : (B ⊸ C) ⊸ (A ⊸ B) ⊸ A ⊸ C & \text {if $g$ is linear too then $x$ is used only once} \\
                    & : ∀ π ρ. (B →_π C) ⊸ (A →_ρ B) →_π A →_{πρ} C & \text{most general type} \\
\end{align*}

The subtyping relation that we have outlined before goes part of the
way.  Yet, if we want the most general types of higher order
functions, we need explicit quantification over weights. Still, this
is somewhat simpler that what Morris presents in ``The best of both
worlds'', because we do not need bounded quantification. Instead we
use multiplication of weights.

\section{Session types vs. linear types}

A good resource explaining session types vs. linear types is Wadler's
``Propositions as Sessions'' (even though it contains some subtle
traps). In sum, session types type 'live' sessions with long-lived
channels, whose type ``evolve'' over time. A contrario, linear types
are well suited to giving types to a given bit of information. One can
see thus that linear types are better suited to a language based on a
lambda calculus, while session types are better suited for languages
based on a pi-calculus and/or languages with effects. Yet; there is a
simple encoding from session types to linear types (as Wadler
demonstrates).



\end{document}
