% Created 2016-09-15 tor 14:09
\documentclass[11pt]{article}
\usepackage[backend=biber,citestyle=authoryear,style=alphabetic]{biblatex}
\bibliography{PaperTools/bibtex/jp.bib}
\usepackage{fixltx2e}
\usepackage{graphicx}
\usepackage{grffile}
\usepackage{longtable}
\usepackage{wrapfig}
\usepackage{rotating}
\usepackage[normalem]{ulem}
\usepackage{amsmath}
\usepackage{textcomp}
\usepackage{amssymb}
\usepackage{capt-of}
\usepackage{hyperref}
\usepackage{mathpartir}
\usepackage{fontspec}
\usepackage{unicode-math}
\setmonofont{DejaVu Sans Mono}
\setmainfont[]{DejaVu Sans}
\newcommand{\case}[2]{\mathsf{case} #1 \mathsf{of} \{#2\}^m_{k=1}}
\newcommand{\inl}{\mathsf{inl} }
\newcommand{\inr}{\mathsf{inr} }
\newcommand{\flet}[1][]{\mathsf{let}_{#1} }
\newcommand{\fin}{ \mathsf{in} }
\newcommand{\susp}[1]{⟦#1⟧}
\author{Jean-Philippe Bernardy and Arnaud Spiwack}
\date{\today}
\title{Linear and unrestricted at the same time}
\hypersetup{
 pdfauthor={Jean-Philippe Bernardy and Arnaud Spiwack},
 pdftitle={Linear and unrestricted at the same time},
 pdfkeywords={},
 pdfsubject={},
 pdflang={English}}
\begin{document}

\maketitle
%% \tableofcontents

\bigskip

In this note we give an overview of a lambda calculus augmented with
weights on variable (bindings). We argue briefly that the calculus is
a generalisation of both:
\begin{itemize}
\item the simply-typed lambda calculus and
\item the linear lambda calculus
\end{itemize}

Additionally, we give
\begin{itemize}
\item an operational semantics which demonstrates how memory
  management of linear values can be simplified.
\item example types for a number of simple programs
\item a brief comparison with alternative approaches to linearity
  (tracking linearity in types; session types)
\end{itemize}

\section{Statics}
\label{sec:orgheadline8}
\subsection{Weights}
\label{sec:orgheadline1}

The set of weights is the two-element set $\{1,ω\}$. Intuitively, 1
represents \emph{exactly one} instance, while $ω$ represents any number of
instances (including zero).  The metasyntactic variables \(π\) and \(ρ\)
range over this set. Weights are equipped with + and $·$ operations:


\begin{align*}
1 · ω = ω \\
ω · 1 = ω \\
ω · ω = ω \\
1 · 1 = 1
\end{align*}

\begin{align*}
1 + ω &= ω \\
ω + 1 &= ω \\
1 + 1 &= ω \\
ω + ω &= ω
\end{align*}

We equip weights we a partial order (written \(· ≤ ·\)) such that \(1 ≤ ω\).

\emph{(An alternative which may be worth pursuing is to choose the set of
weights to be \(ℕ∪{ω}\), in which case \(1+1=2\).)}

\subsection{Types}
\label{sec:orgheadline2}

We propose a variant of the simply typed λ-calculus where the types
are:

\begin{itemize}
\item a weighted arrow: \(A →_π B\) (you can get a \(B\) if you can provide a
quantity \(π\) of \(A\))
In particular:
\begin{itemize}
\item \(A →_ω B\) is the intuitionistic arrow \(A → B\)
\item \(A →_1 B\) is the linear arrow \(A ⊸ B\)
\end{itemize}
\item weighted ADTs:
\begin{align*}
\mathsf{data} D = (c_k  π₁ A₁  …  π_{n_k} A_{n_k})^m_{k=1}
\end{align*}
(\(D\) has \(m\) constructors \(c_k\), for \(k ∈ 1…m\)).

Note the following special cases:
\begin{itemize}
\item The type data Pair a b = Pair ωa ωb is isomorphic to the intuitionistic product (a×b)
\item The type data Tensor a b = Pair 1a 1b is isomorphic to the linear tensor product (a⊗b)
\item The type data Bang a = Box ωa is isomorphic to the exponential (!a)
\end{itemize}
\end{itemize}

The syntax of types and type declarations is as follows ($c$ ranges over constructor names):
\begin{align*}
  W &::= πA &\text{weight-annotated type}\\
  γ &::= c  \vec{W}&\text{constructor declaration}\\
  decl &::= \mathsf{data } D = \vec{γ}&\text{data type declaration}\\
  A,B &::=\\
      & |  A →_π B &\text{function type}\\
      & |  D &\text{data type}
\end{align*}

\subsection{Terms}
\label{sec:orgheadline3}

The syntax for terms is:

\begin{align*}
e,s,t,u & ::= \\
    & |  x & \text{variable} \\
    & |  λx. t & \text{abstraction} \\
    & |  t_π s & \text{application} \text{We sometimes omit the weight $π$ when it's obvious from the context} \\
    & |  c t₁ … t_n & \text{data construction} \\
    & |  \case t {c_k  x₁ … x_{n_k} → u_k}  & \text{case} \\
    & |  \flet x =_{π₁} … x =_{π_n} t_n \fin u & \text{let}
\end{align*}

\subsection{Contexts}
\label{sec:orgheadline6}

Every variable binding in a context is equipped with a weight,
additionally to the type:

\begin{align*}
  Γ,Δ & ::=\\
    & |  x :_π A, Γ & \text{weight-annotated binder} \\
    & |     & \text {empty context}
\end{align*}

\subsubsection{Context addition}
\label{sec:orgheadline4}

We define a context addition, written \(Γ+Δ\), as follows.

\begin{itemize}
\item When a variable appears on both input contexts, we add the weights in
the result:

\((x :_ρ A,Γ) + (x :_π A,Δ) = x :_{ρ+π} A, (Γ+Δ)\)

\item otherwise we propagate:

\((x :_ρ A,Γ) + Δ = x :_ρ A, Γ+Δ\)   \hfill   \(x ∉ Δ\)
\end{itemize}


\subsubsection{Context scaling}
\label{sec:orgheadline5}

Contexts can be scaled by a weight:

\begin{displaymath}
ρ(x :_π A, Γ) =  x :_{πρ} A, ρΓ
\end{displaymath}

\subsection{Typing rules}
\label{sec:orgheadline7}

We have a weighted typing judgement \(Γ ⊢ x:_π A\), inductively defined by the following rules.

\begin{mathpar}
\inferrule{ }{ωΓ + x :_ρ A ⊢ x :_ρ A}\text{variable}

\inferrule{Γ, x :_{πρ} A  ⊢   t :_ρ B}
          {Γ ⊢ λx. t  :_ρ  A  →_π  B}\text{abs}

\inferrule{Γ ⊢ t :_ρ  A →_π B  \\   Δ ⊢ u :_{πρ} A}
          {Γ+Δ ⊢ t u  :_ρ  B}\text{app}

\inferrule{Δᵢ ⊢ tᵢ :_{πᵢρ} Aᵢ \\ \text {$c_k$ is a constructor of $D$ with arguments $Aᵢ$ and weights $πᵢ$}}
          {\sum_i Δᵢ ⊢ c_k  t₁ … t_n :_ρ  D}\text{constructor}

\inferrule{Γ   ⊢  t  :_{ρ} D  \\ Δ, x₁:_{πᵢρ} Aᵢ, …, x_{n_k}:_{π_{n_k}ρ} A_{n_k} ⊢ u_k :_ρ C \\ \text{with each $c_k$ as above}}
          {Γ+Δ ⊢ \case t {c_k  x₁ … x_{n_k} → u_k} :_ρ C}\text{case}

\inferrule{Γᵢ   ⊢  tᵢ  :_{πᵢρ} Aᵢ  \\ Δ, x₁:_{π₁ρ} A₁ …  x_n:_{π_{n}ρ} A_n ⊢ u :_ρ C }
          {\sum_i Γᵢ+Δ ⊢ \flet x =_{π₁} t₁  …  x =_{π_n} t_n  \fin u :_ρ C}\text{let}


\end{mathpar}

Lemmas:
\begin{enumerate}
\item \(πΓ ⊢ t:_{πρ} A\) if \(Γ ⊢ t:_ρ A\).
\item erasure of weights preserves typing
\item unrestricted values need unrestricted contexts

\(Γ ⊢ t :_ω A ~⇒~  ωΓ = Γ \)

Proof: in every rule, if the conclusion has weight $ρ$, then the
premises have weight $πρ$, for some $π$.
\end{enumerate}


\section{Dynamics}
\label{sec:orgheadline16}
\subsection{(Extended) Launchbury semantics}
\label{sec:orgheadline11}

\subsubsection{Translation}
\label{sec:orgheadline9}
As in \textcite{launchbury_natural_1993}, we translate from terms to
terms where values are always bound to variables. 


\begin{align*}
(λx. t)^* &= λx. (t)^* \\
x^*       &= x \\
  (t_π  x )^* &= (t)^*_π  x \\
  (t_π  u )^* &= \flet y =_{π} (u)^* \fin (t)^*_π  y \\
c_k  t₁ … t_n &= \flet x₁ =_{π_1} (t₁)^*,…, x_n =_{π_n} (t_n)^* \fin c_k x₁ … x_n \\
(\case y {c_k  x₁ … x_{n_k} → u_k})^* &= \case y {c_k  x₁ … x_{n_k} → (u_k)^*} \\
(\case t {c_k  x₁ … x_{n_k} → u_k})^* &= \flet y =_1 (t)^* \fin \case y {c_k  x₁ … x_{n_k} → (u_k)^*} \\
(\flet x =_{π₁} t₁  …  x =_{π_n} t_n \fin u)^* & = \flet x₁ =_{π_1} (t₁)^*,…, x_n =_{π_n} (t_n)^* \fin (u)^*
\end{align*}

\subsubsection{Natural semantics}
Compared to Launchbury, we add a linear heap, ranged over by Φ,Ψ,Ξ.
(GC'ed heap is ranged over by Γ,Δ,Θ). Other than this, only the rule
for \emph{variable} and \emph{let} are changed. We also add the
weight: the quantity of values to produce. (It is necessary to do so
because let bindings written statically with weight 1 may endup being
used several times if called from an ω context --- in particular when
these bound variables end up in a data structure.)


\begin{mathpar}
\inferrule{ }{Γ;Φ : λx. e ⇓_ρ Γ;Φ : λx. e}\text{abs}


\inferrule{Γ;Φ : e ⇓_ρ Δ;Ψ : λy.e' \\  Δ;Ψ : e'[x/y] ⇓_{πρ} Θ;Ξ : z}
           {Γ;Ψ : e_π x ⇓_ρ Θ;Ξ : z} \text{application}

\inferrule{Γ;Φ : e  ⇓_ω  Δ;Ψ : z}{(Γ,x ↦ e;Φ) : x ⇓_ρ (Δ;x ↦ z;Ψ) : z}\text{unrestricted variable}


\inferrule{Γ;Φ : e ⇓_1 Δ;Ψ : z}
{(Γ;Φ,x ↦ e) : x ⇓_1 Δ;Ψ : z}\text{linear variable}


\inferrule{(Γ,x_i ↦ e_i;Φ,x_j ↦ e_j) : e ⇓_ρ Δ;Ψ : z}
{Γ,Φ : \flet x₁ =_{π₁} e₁ … x_n =_{π_n} e_n \fin e ⇓_ρ Δ;Ψ : z}\text{let}

\text{where $i$ ranges over the subset of indices such that $πᵢρ$ = ω,}

\text{and $j$ ranges over the subset such that $π_jρ$ = 1.}


\inferrule{ }{Γ;Φ : c  x₁ … x_n ⇓_ρ Γ;Φ : c  x₁ … x_n}\text{constructor}


\inferrule{Γ;Φ: e ⇓_ρ Δ,Ψ: c_k  x₁ … x_n \\   Δ,Ψ :  e_k[xᵢ/yᵢ] ⇓_ρ Θ,Ξ :  z}
   {Γ;Φ :  \case e {c_k  y₁ … y_n ↦ e_k } ⇓_ρ Θ,Ξ :  z}\text{case}
\end{mathpar}

Remark: the \emph{unrestricted variable} rule also triggers when the
weight is 1, thus effectively potentially triggering for linear
variables. This behaviour allows an occurence of a linear variable to
work in an unrestricted contexts, in turn justifying the $1 + ω = ω$
rule.


\paragraph{Theorem: The unrestricted heap contains no references to the linear heap}
Proof: TODO.  (Uses simultaneously the property that when the weight
is ω, then we have that the produced value has no reference to the
linear heap)

(This result is critical for heap consistency.)

Yet, the following example may, at first glance, look like a counter
example where \verb|x| is in the non-GC heap while \verb|y| in the
GC-heap points to \verb|x|:
\begin{verbatim}
data () = ()

let x =_1 ()
let y =_ω ( case x of { () -> () })
in ()
\end{verbatim}
However, while \verb|()| can indeed be typed as $⊢ () :_ω ()$, the
case rules gives the same weight to the case-expression than to the
scrutinee (\verb|x| in this case). Therefore
\verb|case x of { () -> ()}| has weight 1.

Remark: for a program to turn a 1-weight into an ω-weight, one may use
the following definition:
\begin{verbatim}
data Bang A = Box ωA
\end{verbatim}
The expression \verb|case x of { () -> Box ()}| has type
\verb|Bang A|, but still with weight 1.  This pattern does not apply
just to the unit type $()$, but to any data type \verb|D|. Indeed, for such
a type we will have a function \verb|D ⊸ Bang D| (this may be even
efficiently implemented by copying a single pointer --- for example if
we have a single array, or a notion of compact region).  Thus at any
point where we have an intermediate result comprised of data only, we
may switch to use the linear heap. In a second phase, this data may
then be moved to the GC heap and used for general consumption.

In that light, the only way to use a linear value from the GC-heap is
to force it first, and then chain computations with \verb|case| --- for
example as follows:
\begin{verbatim}
let x =_1 ()
case ( case x of { () -> Box () }) of {
  Box y -> ()
}
\end{verbatim}
This still does not create a pointer from GC-heap to non-GC heap: by the
time \verb|y| is created, the linear value \verb|x| has been freed.

If, on the other hand, \verb|x| had weight $ω$, then we would be in the
usual Haskell case, and the following expression does type:
\begin{verbatim}
let x =_ω ()
let y =_ω ( case x of { () -> () } )
in ()
\end{verbatim}


\subsection{{\bfseries\sffamily TODO} KAM}
\label{sec:orgheadline14}

Skip this section: it has not yet been retrofitted with the syntax above.

The dynamic semantics is given as a variant of Krivine's abstract
machine, modified to support two environments. One environment is
unrestricted, while the other is linear.

\begin{itemize}
\item As in Krivine's original, the abstract machine has a call-by-name
evaluation strategy, not true laziness (that is, sharing of the
intermediate results). This is sufficient to demonstrate the
principles of linear logic memory management.

\item We assume that the input terms are annotated with:
\begin{itemize}
\item weights (for variables)
\item how the context is split (for terms where this happens)
\end{itemize}

\item The machine runs the same way in ω or 1 context. This means that we
have 'weight polymorphism' at runtime. (The same code can run in
linear or unrestricted environment.)
\end{itemize}

\begin{align*}
state       &::= (t,σ,φ,ψ) & \text{machine state} \\
ξ     &::= (t,φ,ψ) \mid \susp{t,φ,ψ} & \text{closure} \\
σ       &::= [] \mid ξ:σ & \text{stack} \\
φ &::= [] \mid (x↦ξ,φ) & \text{environment} \\
ψ &::= [] \mid (x↦ξ,φ) & \text{linear environment}
\end{align*}

\begin{itemize}
\item We write \(ψ/Γ\) for the environment restricted to the variables and
weights in \(Γ\). We have the following property: if \(Γ = Δ₁+Δ₂\) and
\(ψ : Γ\), then \(ψ = ψ/Δ₁ ⊎ ψ/Δ₂\) (because \(ψ\) is a linear
environment, \(ψ/Δ₁\) and \(ψ/Δ₂\) have disjoint support).
\end{itemize}

\hspace{-4cm}\begin{minipage}{\textwidth}
\begin{align*}
(u_Γ t_Δ,σ,φ,ψ)                                                  &⟶ (u,((t,φ,ψ/Δ):σ),φ,ψ/Γ)\\
(λx_ω.t,ξ:σ,φ,ψ)                                                    &⟶ (t,σ,((x↦ξ),φ),ψ)\\
(λx_1.t,ξ:σ,φ,ψ)                                                    &⟶ (t,σ,φ,(x↦ξ),ψ)\\
(x_1,σ,\_,[x↦(t,φ,ψ)])                                               &⟶ (t,σ,φ,ψ)\\
(x_ω,σ,(x↦(t,φ,ψ):\_),[])                                            &⟶ (t,σ,φ,ψ)\\
(\inl t,\susp{\case □ {\inl x → u; \inr y → u'}:σ,φ',ψ'},φ,ψ) &⟶ (u,σ,φ',(x↦(t,φ,ψ)):ψ')\\
(\case t {\inl x → u; \inr y → u'},σ,φ,ψ)                        &⟶ (t, \susp{\case □ {\inl x → u; \inr y → u'},φ,ψ/Γ}:σ,φ,ψ/Δ)\\
(\flet (x_π,y_ρ) = t \fin u,σ,φ,ψ)                                       &⟶ (t, \flet (x_π,y_ρ) = □ \fin u,φ,ψ/Γ):σ,φ,ψ/Δ)\\
((t_Γ,u_Δ),\susp{\flet (x_π,y_ρ) = □ \fin v:σ,φ',ψ'},φ,ψ')         &⟶ (v,σ,φ', (x↦(t,φ,ψ/Γ)),(y↦(u,φ,ψ/Γ)),ψ')
\end{align*}
\end{minipage}

\subsubsection{Theorems}
\label{sec:orgheadline12}

\begin{itemize}
\item If \(Γ ⊢ t :_ρ A\), then \((t,[],[],[]) ⟶^* (u,\_,\_,\_)\) where \(u\) is in whnf.
\item No part of the linear environment is ever duplicated (or shared).
\end{itemize}

\subsubsection{Remarks}
\label{sec:orgheadline13}

\begin{itemize}
\item In the reduction \((u_Γ t_Δ,σ,φ,ψ)\), one may fear that we're
putting a linear part of the environment \(Δ\) in a closure, which will
end up being shared afterwards. The system is engineered so that
this case cannot occur. Indeed
\begin{itemize}
\item when the weight of \(t\) is \(ω\), then \(Δ\) contains only \(ω\) weights; the
linear part of \(Δ\) is empty, so is \(ψ/Δ\):

\((u_Γ t_{ωΔ},σ,φ,ψ)          ⟶ (u,((t,φ,[]):σ),φ,ψ)\)

\item when the weight of \(t\) is 1, then the closure will only be consumed
by a linear λ --- it will not be duplicated.
\end{itemize}

\item Consequently, for every closure in the unrestricted environment, the
linear environment embedded in the closure is empty.
\end{itemize}


\subsection{Examples of simple programs and their types}

\hspace{-4cm}\begin{minipage}{\textwidth}
\begin{align*}
map & : (A ⊸ B) → [A] ⊸ [B] & \text{Scales well in unrestricted contexts}\\
lmap & : (A ⊸ B) → f A ⊸ f B & \text{Guaranteeing that no element is lost} \\
λx. λy. x & : A ⊸ B → A \\
          & : A → B → A \\
λ(f,x). f x & : \{f :_1 A ⊸ B, x :_1 A\} ⊸ B & \text{(using a row type for concision)}\\
            & : \{f :_1 A → B, x :_ω A\} ⊸ B \\
            & : \{f :_1 A →_π B, x :_π A\} ⊸ B & \text{most general type} \\ 
λf. λg. λx. f (g x) & : (B → C) → (A → B) → A → C \\
                    & : (B → C) ⊸ (A → B) → A → C & \text {always better: comp uses $f$  only once} \\
                    & : (B ⊸ C) ⊸ (A → B) ⊸ A → C & \text {if $f$ is linear, then $g$ is used only once} \\
                    & : (B ⊸ C) ⊸ (A ⊸ B) ⊸ A ⊸ C & \text {if $g$ is linear too then $x$ is used only once} \\
                    & : ∀ π ρ. (B →_π C) ⊸ (A →_ρ B) →_π A →_{πρ} C & \text{most general type} \\
\end{align*}
\end{minipage}

If we want the most general types of higher order functions, we need
explicit quantification over weights. Still, this is somewhat simpler
than what Morris presents in ``The best of both worlds'', because we
do not need bounded quantification. Instead we use multiplication of
weights. (There is also the possibility to add a subtyping relation as
a middle ground.)

\section{Linearity as a property of types vs. linearity as a property of bindings (variables)}

In several presentations (\cite{wadler_linear_1990,mazurak_lightweight_2010,morris_best_2016}
programming languages incorporate
linearity by dividing types into two kinds. A type is either linear
or unrestricted. Unrestricted types typically includes primitive types
(\texttt{Int}), and all (strictly positive) data types. Linear types
include typically resources, effects, etc.

A characteristic of this presentation is that linearity ``infects''
every type containing a linear type. Consequently, if we want to make
a pair of (say) an integer and an effect, the resulting type must be
linear. This property means that polymorphic data structure can no
longer be used ``as is'' to store linear values. Technically, one
cannot unify a type variable of unrestricted kind to a linear
type. One can escape the issue by having polymorphism over kinds;
unfortunately to get principal types one must have subtyping between
kinds and bounded polymorphism.

In contrast, we have automatic scaling of linear types to unrestricted
ones in unrestricted contexts. This feature already partially
addresses the problem of explosion of types. In order to get principal
types we need quantification over weights, and extension of the
language of weights to products and sums.

Another issue with the ``linearity in types'' presentation is that it
is awkward at addressing the problem of ``simplified memory
management'' that we aim to tackle. As we have seen, the ability to
use an intermediate linear heap rests on the ability to turn a linear
value into an unrestricted one. When linearity is captured in types,
we must have two versions of every type that we intend to move between
the heaps. Even though it is possible to provide this, it is somewhat
annoying to duplicate every primitive type. (Possibly we could
prescribe \#Int to be linear, but this may break lots of existing
programs.)

\section{Session types vs. linear types}

\Textcite{wadler_propositions_2012} provides a good explanation of
the relation between session types vs. linear types (even though the
paper contains some subtle traps --- notably the explanation of par
and tensor in LL does not match the semantics given later.). In sum,
session types classify 'live' sessions with long-lived channels, whose
type ``evolves'' over time. In contrast, linear types are well suited
to giving types to a given bit of information. One can see thus that
linear types are better suited for a language based on a lambda
calculus, while session types are better suited for languages based on
a pi-calculus and/or languages with effects. Or put another way,
it's a matter of use cases: session types are particularly well-suited
to naturally describe communication protocols, while linear types are
well suited for describing data. One is communication centric. The
other is data centric. Yet there is a simple
encoding from session types to linear types (as Wadler demonstrates).

\printbibliography
\end{document}
